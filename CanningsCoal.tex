\documentclass[review,nonatbib]{elsarticle}

\usepackage[]{geometry}
\usepackage{amsmath}
\usepackage{amsthm}
\usepackage{amssymb}
\usepackage{graphicx}
\usepackage{xcolor}
\usepackage{hyperref}
\usepackage{float}
% biblatex + elsarticle fix
\makeatletter
\let\c@author\relax
\makeatother
\usepackage[natbib=true,citestyle=authoryear]{biblatex}
\addbibresource{taming-strong-selection.biblatex}

\newcommand{\ra}{\rightarrow}
\newcommand{\afs}[2]{\Phi_{#1}^{(#2)}}
\newcommand{\Dfrac}[2]{%
  \ooalign{%
    $\genfrac{}{}{1.2pt}0{#1}{#2}$\cr%
    $\color{white}\genfrac{}{}{.4pt}0{\phantom{#1}}{\phantom{#2}}$}%
}
\newcommand{\cond}{\middle\vert}
\newcommand{\dslash}{/\!\!/}
\newcommand{\Coalc}[4]{\begin{bmatrix}#1\dslash #2 \\ #3\dslash #4 \end{bmatrix}}

\newcommand{\CC}{\mathcal{C}}
\newcommand{\ms}{\mathcal{S}}
\newcommand{\QQ}{\mathcal{Q}}

\newcommand{\hypo}{\operatorname{hypo}}


\newcommand{\sgcomment}[1]{{\color{red}{SG: #1}}}
\newcommand{\ikcomment}[1]{{\color{blue}{IK: #1}}}
\newcommand{\Var}{\operatorname{Var}}
\newtheorem{proposition}{Proposition}

\journal{Theoretical Population Biology}

\begin{document}
\begin{frontmatter}
  \title{ Family sizes in coalescent world }

  \author{Ivan Krukov}
  \author{Simon Gravel}

  \begin{abstract}
  The collection of large genetic datasets has recently put into question the validity of classical population genetic approximations, 
  notably Kingman's coalescent and the diffusion approximation, which can be thought of as approximations to the Wright-Fisher model appropriate 
  in the small sample size limit.
   Differences in diversity statistics predicted by these models has a concrete bearing on the inference of demographic
    and selective parameters. 

Such differences have been tied to the presence (or absence) of simultaneous mergers 
in gene genealogies of the two models. 

Here we argue that many differences between these models stem from the large differences in family sizes between Kingman's coalescent and the 
Wright-Fisher model, where Kingman's coalescent implies a high number of very large families. 
 
 Thus the `best' model depends not on mathematical properties of each model, but on the demographic realism of the underlying 
 assumptions about family sizes.      
  
  
  \end{abstract}

\end{frontmatter}


\section{Intro}

\subsection{Family and sibship sizes in the Wright-Fisher model}
The number of offspring carried by an individual in a neutral haploid Wright-Fisher model is binomially distributed and accurately approximated as a Poisson distribution.
The probability that a set of $k$ individuals sampled from a population are part of a sibship is simply $P(\tau_k)=\frac{1}{N^{k-1}}.$ The probability that the same 
$k$ individuals form a complete sibship (i.e., that no one else is part of the sibship) is $P(\sigma_k) = \frac{1}{N^{k-1}} \left(1-\frac{1}{N}\right)^{N-k} \simeq \frac{1}{e N^{k-1}},$ where
the last approximation supposes $k\ll N$.


\section{Family sizes in the coalescent}

Continuous coalescent models do not feature discrete generations, and therefore do not have an explicit notion of sibships or families. 
However, a natural definition of a sibship, in a haploid population, is a set of individuals who share a common ancestor exactly one generation ago.

The probability that $k$ individuals are part of a sibship is easily expressed in terms of the 
cumulative distribution of the hypoexponential distribution: $P(\tau_k) = CDF(\hypo(\{\lambda_i\}_{i\in\{2,\ldots,k\}}),1),$ where $\lambda_i=\frac{i (i-1)}{2N}$ is 
the rate of coalescence among $i$ lineages. 

\begin{proposition}
 $P(\tau_k) =  \frac{ k! +O(N)^{-1}}{2^{k-1}N^{k-1}}.$ 
 \end{proposition}

\begin{proof}
We have $P(\tau_1)=1$. Now assume that $P(\tau_{k-1}) =  \frac{ (k-1)! +O(N)^{-1}}{2^{k-2}N^{k-2}}.$ We can condition
on the waiting time of the first coalescent event, and denote $P_{1-\epsilon}(\tau_k)$ as the probability that $k$ lineages coalesce within time 
$1-\epsilon.$ In Kingman's coalescent, we can rescale times and population sizes to maintain probability:
 $P_{1-\epsilon}(\tau_k) =  \frac{ (k-1)! +O(N)^{-1}}{2^{k-2}\left(\frac{N}{1-t}\right)^{k-2}}.$ 

We can therefore write
\begin{equation}
\begin{split}
P(\tau_k) &= \int dt_k P(t_k) P_{1-\epsilon}(\tau_{k-1}) \\
&= \int dt_k \lambda_k \exp(-\lambda_k t)  \frac{ (k-1)! +O(N)^{-1}}{2^{k-2}\left(\frac{N}{1-t}\right)^{k-2}}  \\ 
&= \int dt_k \lambda_k\left(1-O(N)^{-1}\right)  \frac{ (k-1)! +O(N)^{-1}}{2^{k-2}\left(\frac{N}{1-t}\right)^{k-2}}  \\
&= \frac{ k! +O(N)^{-1}}{2^{k-1}N^{k-1}} \qedhere
\end{split}
\end{equation}
 \end{proof}
 
    

With the exception of $k=2,$ the probability that sampled individuals are part of a kinship,(i.e., that we see multiple mergers within one generation) is much larger than under the Wright-Fisher model (Figure X), and produces large departures even for modest $k$. 
For example,  the probability of finding a three-way merger over the course of a single generation is three times larger in a coalescent model compared to the 
Wright-Fisher model. 


\section{Mathematical intuition}
The excess of large sibships in coalescent models can be interpreted from a conditional sampling perspective.  If $\tau_k$ is the event that the 
first $k$ sampled individuals have a common ancestor, then in both models $P(\tau_k) =  P(\tau_k|\tau_{k-1}) P(\tau_{k-1}) = \prod_{i=2}^k P(\tau_i|\tau_{i-1}).$ In the
 WF model, $P(\tau_i|\tau_{i-1}) = \frac{1}{N}.$ In the coalescent, the probability of coalescing to the previous $i-1$ previously coalesced lineages is approximately equal to 
$\frac{T_{i-1}}{N},$ where $T_{i-1}$ is the length of the tree formed by the previously coalesced lineages. In particular, $T_{i-1}>T_{1}=1,$ so that every term after the first one
 in the product for $P(\tau_k)$ is larger (and possibly much larger) in the coalescent relative to WF.  This reinforcement effect allows for the buildup of
 large sibships relative to the Wright-Fisher model. 
 
 A perhaps surprising corollary of $P(\tau_k) =  P(\tau_k|\tau_{k-1}) P(\tau_{k-1})$ is that $P(\tau_k|\tau_{k-1})= \frac{k+O(1/N)}{2 N},$ 
 and therefore that $T_k = 1+(k-1)/2,$ i.e., the size of the tree, conditional on coalescence,
  grows linearly in the sample size. \sgcomment{This was confusing to me at first, but it does follow from the proposition. I feel like there must be an intuitive argument here, but can't find it}
 
 Even though the number of large sibships in the coalescent is much larger than in WF,  the overall coalescence rate is lower, according to \cite{Bhaskar}. 
 This must entail that the number of coalescences attributed to small families in WF is higher than in the coalescent. As an example, consider the probability 
 that the first $2k$ lineages are chosen such that lineage $2i$ is paired only with lineage $2i-1$ for $i \in \{1,\ldots,k\}.$ Because of the added length 
 trees formed by the previous pairs, these are attractive to our new lineages and reduce the overall probability of having only two-way coalescent events.
 \sgcomment{This is correct and could be formalized, but a fairly weak argument. I'd rather show concretely that the overall rate is reduced. 
 Maybe it can be done with a lookdown construction.}  So overall, the coalescent is just more likely to have very large families, not more likely to coalesce overall (?!). 
  
  
 Conversely, the effect on the total rate of coalescence is opposite. If we ask for the expected number of coalescences $C_k$ 
 within a sample of size $k$, we get $C_k= \sum_{i=1}^k  c_i$, with $c_i$ the probability that lineage $i$ coalesces in any of the previous lineages. 







\end{document}
